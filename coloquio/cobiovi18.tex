\documentclass[a4paper,9pt]{article}
\usepackage[latin1]{inputenc}
\usepackage[brazil]{babel}
\usepackage{epsfig,float,graphicx,graphics,amssymb,amsfonts,newlfont,indentfirst}
\usepackage[centertags]{amsmath}
\usepackage{fancyhdr}
\usepackage{ragged2e}
\usepackage{geometry}
\geometry{top=3cm,bottom=2cm,left=3cm,right=2cm}
\usepackage[normalem]{ulem}
\pagestyle{fancy}

\headheight 30mm      %
\oddsidemargin 2.0mm  %
\evensidemargin 2.0mm %
\topmargin -1mm       %
\textheight 215mm     %
\textwidth 160mm      %
\headsep 2mm        %
\parindent 1mm        %


\lhead{\epsfig{file=LogoCOBIOVI.JPG, height=0.05cm, width=0.05cm}}
\vskip 0.2cm
\chead{
{\Large{ \bf COBIOVI 2018}}\\
{ {\bf {I COLOQUIO INTERNACIONAL DE BIOMETRIA PARA LAS CIENCIAS DE LA VIDA}}}  \\ {5 a 7 de Diciembre 2018} \\
{\small UNMSM, Lima -- Per�} 
\\}
\rhead{I COBIOVI
%\epsfig{file=Logo.jpg, height=0.5cm, width=0.5cm}
}

\cfoot{\thepage}
\renewcommand{\baselinestretch}{0.35}

\begin{document}
\centering{
\renewcommand{\baselinestretch}{1.0} \Large
{\Large{\bf Instrucciones para la preparaci�n de trabajos que ser�n presentados en COBIOVI 2018}}

\renewcommand{\baselinestretch}{1} \tiny \normalsize

\vspace*{5mm}
 {\bf {\large Primer Autor }}  \hspace*{1cm} {\bf {\large Segundo Autor}},  \\
 {\small Departamento de Matem�ticas, FCM, UNMSM} \\
 {\small Cusco, Per�} \\
 {\small E-mail: autor1@uni.edu.pe, \hspace*{.2cm}  autor2@pucp.edu.pe} \\

\vspace*{5mm}
 {\bf {\large Tercer Autor}}  \\
 {\small FCM, UNMSM} \\
 {\small Cercado de Lima,  Lima, Per�}\\
 {\small E-mail: autor3@unmsm.edu.pe}\\
}
\vspace*{5mm}


\setcounter{equation}{0}

\justifying

\section*{Abstract}
\quad
\justifying
Estimado colega, use esta plantilla para elaborar  su resumen para el Congreso Latinoamericano de Biomatem\'atica. 

\section*{Keywords}
\justifying
Usar a lo m�s 6 palabras-claves.

\section*{Introducci�n}
\quad
\justifying
El resumen debe tener a lo  mas cuatro p\'aginas.
Los trabajos enviados para presentaci�n en
COBIOVI 2018 pueden ser escritos en espa�ol o portugu\^es o ingles, siguiendo esta plantilla.  En la Tabla \ref{t1} hay un modelo de tabla y una Figura \ref{f1} un ejemplo de inclusi�n de figura.


\begin{table}[H]
\centering{\caption{ Ejemplo de tabla.}\label{t1}
\begin{tabular}{cccc}  \hline
Nomes & Librerias & F\'ormulas &Libros \\
\hline
P\'assaro        & Ot & $\int x \,dx$  & Alberto\\
\'Arvore       & Bv & $x-\exp(x)$& Britton \\
Flores    & Blacks & $X_{3}$  & Rodney    \\
Chuva & Smith & $\frac{1}{3} $   & Murray     \\\hline
\end{tabular}}
\end{table}

\justifying
El archivo con el trabajo  debe ser enviado en PDF y TEX. El periodo para el envio del trabajo es:
\begin{itemize}
\item
 {\bf { de 10/07/2018 al 10/10/2018.}} 
\end{itemize}


Otras informaciones:

{\small{{\bf { https://sites.google.com/a/unmsm.edu.pe/gi_cmatvida/


rlopezc@unmsm.edu.pe}}}}

%\begin{figure}[H]
%\centering \epsfig{file=f4.eps,width=4cm}
% \caption{ { Curva de nivel}.}
%\label{f1}
%\end{figure}

\justifying
Las referencias bibliogr\'aficas deben  colocarse como referencias tipo APA. Como ejemplo, la referencia bibliogr\'afica \cite{chap1} es para articulo en revista,  \cite{mend} es para trabajos de conclus��n de programa (pregrado, maestria, doctorado), e \cite{wein} para libro.

\section*{Agradecimentos}
\quad
\justifying
 Coloque aqui los agradecimientos.

\begin{thebibliography}{00}
\bibitem{chap1}
M. A. J. Chaplain, S. R. McDougall, A. R. A. Anderson, \newblock \textit{Mathematical modeling of
tumor-induced angiogenesis,} Annu Rev Biomed Eng, 8, 233--257, 2006.

\bibitem{mend}
G. Olic\'on M\'endez, \newblock\textit{Bifurcaciones secundarias y la bifurcaci\'on de Hopf--Turing}. Tesis (T\'itulo de Matem\'atico), Facultad de Ciencias, UNAM, 2013.

\bibitem{wein}
R. A. Weinberg,\newblock \textit{The biology of cancer}, New York, Garland Science, 2008.

\end{thebibliography}
\end{document}

