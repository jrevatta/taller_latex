%%\pagenumbering{roman} \setcounter{page}{1}
%\chapter*{Resumen\markboth{Resumen}{Resumen}}
%Vendan bien sus tesis
\begin{center}
ABSTRACT\\[1cm]
MODELING OF A LINEAR THERMOELASTIC SYSTEM AND ITS COMPUTATIONAL SIMULATION USING PYTHON\\[0.7cm]
%DEDUCTION AND STUDY OF A THERMOELASTIC SYSTEM AND\\ HIS IMPLEMENTATION BASED ON FREE SOFTWARE\\[0.7cm]
LUIS ALONSO MANSILLA ALVAREZ\\[0.7cm]
DECEMBER - 2011\\[0.7cm]
\end{center}

\hspace{-0.6cm}Advisor: \qquad\qquad\qquad\,\,\, DSc. Jos� Raul Luyo S�nchez\\
Obtained Title: \qquad\qquad\,Degree in Scientific Computing\\
\noindent{.\dotfill{}.\par}
\hspace{-0.6cm}We present a complete study of the linear thermoelasticity in one-dimensional case. Based on physical laws, we develope a mathematical model 
which describes this phenomenon based on a system of partial differential equations involving the spatial displacement and temperature.\\
\hspace{-0.6cm}Once retrieved the mathematical model, we obtain the results that guarantee that it is a ``well-posed'' problem in the sense proposed by 
Hadamard, for this, we ensure the existence, uniqueness and regularity of the solution based on the theory of semigroups of operators applied to partial 
differential equations.\\
\hspace{-0.6cm}As central point, we develop a numerical scheme to simulate the behaivor of the phenomenon and propose to use the programming language Python as 
an useful and effective free software for computational implementation and achivement of results by the numerical scheme.\\[1cm]
\begin{tabular}{ l l }
  \hspace{-0.2cm}KEYWORDS: & PARTIAL DIFFERENTIAL EQUATIONS\\
  & THERMOELASTICITY \\
  & SEMIGROUPS \\
  & PYTHON \\
  & ALGORITHMS\\
  
\end{tabular}