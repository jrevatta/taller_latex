%%\pagenumbering{roman} \setcounter{page}{1}
%\chapter*{Resumen\markboth{Resumen}{Resumen}}
%Vendan bien sus tesis
\begin{center}
RESUMEN\\[0.7cm]
MODELAMIENTO DE UN SISTEMA TERMOEL�STICO UNIDIMENSIONAL Y SU SIMULACI�N COMPUTACIONAL USANDO PYTHON\\[0.7cm]
%DEDUCCI�N Y ESTUDIO DE UN SISTEMA TERMOEL�STICO\\ E IMPLEMENTACI�N EN BASE A SOFTWARE LIBRE\\[0.7cm]
LUIS ALONSO MANSILLA ALVAREZ\\[0.7cm]
DICIEMBRE - 2011\\[0.7cm]
\end{center}

\hspace{-0.6cm}Orientador: \qquad\qquad\qquad DSc. Jos� Raul Luyo S�nchez\\
T�tulo obtenido: \qquad\qquad\,Licenciado en Computaci�n Cient�fica\\
\noindent{.\dotfill{}.\par}
%--------------------------------------------------------------------------\\
\hspace{-0.6cm}En este trabajo presentamos un estudio completo de la termoelasticidad lineal en el caso unidimensional. Basados en leyes f�sicas, 
desarrollamos un modelo matem�tico que describa dicho fen�meno en base a un sistema de ecuaciones diferenciales parciales que involucren el desplazamiento
espacial y la temperatura.\\
\hspace{-0.6cm}Planteado el modelo matem�tico, obtenemos los resultados que garantizan que el problema est� ``bien puesto'' en el sentido que propuso 
Hadamard, aseguramos para esto la existencia, unicidad y regularidad de la soluci�n en base a la teor�a de semigrupos de operadores aplicada a las 
ecuaciones diferenciales parciales.\\
\hspace{-0.6cm}Como punto central, desarrollamos un esquema num�rico para simular el comportamiento del fen�meno y proponemos el uso del lenguaje de 
programaci�n Python como tecnolog�a de software libre, �til y efectiva para la implementaci�n computacional y obtenci�n de resultados mediante el esquema 
num�rico. \\[1cm]
\begin{tabular}{ l l }
  \hspace{-0.2cm}PALABRAS CLAVES: & ECUACIONES DIFERENCIALES PARCIALES\\
  & TERMOELASTICIDAD \\
  & SEMIGRUPOS \\
  & PYTHON \\
  & ALGORITMOS\\
  
\end{tabular}