\chapter{PRODUCTO VECTORIAL}
%{\ref{producto.vectorial} \sc Producto vectorial}

\section{ Producto vectorial.}
\label{producto.vectorial}

\bigskip
Consideremos la terna ordenada de vectores de $V$, $\{
\overrightarrow{i} ,\overrightarrow{j} ,\overrightarrow{k} \}$,
que constituye una base de $V$. Estas las vamos a clasificar en
dos grupos: uno formado por las ternas $\left\{ {i^{} ,j,k }
\right\}$ tales que para un observador parado en $v_3$, la
rotaci\'{o}n de \'{a}ngulo convexo (o sea, de medida $< \pi $) que
lleva $i$ en $j$ es de sentido antihorario; el otro formado por
las ternas en las que esa rotaci\'{o}n tiene sentido horario.
% Warning: EPS-printer was not specified  -- Figure omitted!


\begin{figure}[htb]
\centering
FIGURA
%\includegraphics[scale=0.4]{orienta.eps}
\end{figure}

\bigskip

Para definir el producto vectorial necesitamos elegir uno de los
dos tipos como terna preferida. Nos quedaremos para esto con las
ternas del primer tipo que llamaremos \textbf {positivas}.
\bigskip

Esta convenci\'{o}n sirve para definir el producto vectorial como
una funci\'{o}n de $V \times V\,\,en\,\,V$ (as\'{\i} como el
producto escalar era una funci\'{o}n de $V \times
V\,\,\,en\,\,\,\,\mathbb{R}$) que a cada par de vectores $v,w$ le
asocia un vector, que denotaremos $v$ $ \wedge $ $w$, que cumple:

\noindent a) $\left\| {v \wedge w} \right\| = \left\| {v}
\right\|.\left\| {w} \right\|.sen\,\theta $  ($\theta $ \'{a}ngulo
de \textit{v} \,\, con \,\, \textit{w} )

\noindent b) $\left( {v \wedge w} \right).v = 0\,\,y\,\,\left( {v
\wedge w} \right).w = 0$

\noindent c) Si $v \ne 0\,\,\,\,y\,\,\,\,w \ne 0$ la terna
$\left\{ {v,w,v \wedge w} \right\}$ es positiva.


\bigskip

\textbf{Propiedades:}

\textbf{1)} $\left\| {v \wedge w} \right\|$ es el doble del
\'{a}rea del tri\'{a}ngulo determinado por esos vectores.


\textbf{2)} $v \wedge w = 0$ si y s\'{o}lo si \textit{v }y w son
colineales (en particular, si alguno de ellos es el vector $0$).


\textbf{3) }Respecto de la condici\'{o}n  c) de la definici\'{o}n,
corresponde observar que si $v \wedge w \ne 0$ la terna $\left\{
{v,w,v \wedge w} \right\}$ es una base, pues por b) esos vectores
no son coplanares salvo que \textit{v }y w sean colineales, en
cuyo caso $v \wedge w = 0$

\textbf{4) }$v \wedge w =  - \left( {w \wedge v} \right)$ ( en
particular , esta operaci\'{o}n no es conmutativa) Para verificar
esto basta notar que si para cierto observador la rotaci\'{o}n del
\'{a}ngulo $ < \pi $ que lleva \textit{v }en w es de sentido
antihorario, para el mismo observador la que lleva w en \textit{v
}es de sentido horario. De modo que el observador debe ubicarse
del lado opuesto del plano u,w para que esa rotaci\'{o}n aparezca
de sentido trigonom\'{e}trico . Luego esa debe ser la
ubicaci\'{o}n de $w \wedge v$ para que la terna $\left[ {w,v,w
\wedge v} \right]$ sea positiva.

\begin{figure}[htb]
\centering
FIGURA
%\includegraphics[scale=0.5]{vectorial.eps}
\end{figure}
%\begin{center}
%% Warning: EPS-printer was not specified  -- Figure omitted!
%
%\end{center}
\textbf{5) }Este producto no es asociativo. Se puede deducir que:
\[
\left( {u \wedge v} \right) \wedge w + \left( {v \wedge w} \right)
\wedge u + \left( {w \wedge u} \right) \wedge v = 0
\] de donde, usando 4), $\left( {u \wedge v} \right) \wedge w = u
\wedge \left( {v \wedge w} \right) + v \wedge \left( {w \wedge u}
\right)$, y como en general $v \wedge \left( {w \wedge u} \right)
\ne 0$, la propiedad asociativa no vale.

\textbf{6) }Para todo $\lambda  \in \mathbb{R}$: $\lambda \left(
{v \wedge w} \right) = \left( {\lambda v} \right) \wedge w = v
\wedge \left( {\lambda w} \right)$. Esta propiedad se deduce
directamente de la definici\'{o}n en el caso $\lambda  \ge
0$. Para el caso $\lambda  < 0$, hay que observar que del hecho
que la terna $\left\{ {v,w,v \wedge w} \right\}$ es positiva se
deduce que $\left\{ { - v,w, - \left( {v \wedge w} \right)}
\right\} $ y
\\ $ \left\{ {v, - w, - \left( {v \wedge w} \right)} \right\}$ son
tambi\'{e}n positivas.

\textbf{7) }El producto vectorial es distributivo respecto de la
suma de vectores. Esto es:

\noindent a) $\left( {v_1  + v_2 } \right) \wedge w = v_1 \wedge w
+ v_{2}^{}  \wedge w.$


\noindent b) $v \wedge \left( {w_{1}^{}  + w_{2}^{} } \right) = v
\wedge w_{1}^{}  + v \wedge w_{2}^{}. $

\noindent Para demostrar esta propiedad hacemos algunas
observaciones previas. En primer lugar, observamos que llamando
$\pi $ al plano con vector normal \textit{v }e indicando por $p $
\textit{w} la proyecci\'{o}n de \textit{w} sobre ese plano, y con
$r(p\textit{w})$ el vector que se obtiene rotando esa
proyecci\'{o}n un \'{a}ngulo de medida $\pi /2$ en sentido
antihorario observado desde $v$, se verifica que: $v \wedge w =
\left\| {v} \right\|.r\left( {pw} \right)$ pues
 $\left\| {r\left( {pw} \right)} \right\| = \left\| {pw} \right\| = \left\|
{w} \right\|.sen\,\theta $ (ver figura)

\begin{figure}[htb]
\centering
FIGURA
%\includegraphics[scale=0.4]{propvec.eps}
\end{figure}
%\begin{center}
%% Warning: EPS-printer was not specified  -- Figure omitted!
%
%\end{center}


\bigskip

%\begin{center}
%FIGURA 4.1
%\end{center}


\bigskip

Como $p\left( {w_{1}^{}  + w_{2}^{} } \right) = pw_{1}^{}  +
pw_{2}^{} $ y $r\left( {pw_{1}^{}  + pw_{2}^{} } \right) = r\left(
{pw_{1}^{} } \right) + r\left( {pw_{2}^{} } \right)$ tendremos
que: $v \wedge \left( {w_{1}^{} \wedge w_{2}^{} } \right) =
\left\| {v} \right\|.r\left( {p\left( {w_{1}^{} + w_{2}^{} }
\right)} \right)$$ = \left\| {v} \right\|.r\left( {pw_{1}^{} }
\right) + \left\| {v} \right\|.r\left( {pw_{2}^{} } \right) = v
\wedge w_{1}^{}  + v \wedge w_{2}^{} $.


As\'{\i} se prueba b) y an\'{a}logamente se obtiene a).

\textbf{8) }Si $ \{i,j,k\}$ es una base ortonormal positiva de $V$
, entonces:
\[
{\begin{array}{l}
 i \wedge i = j \wedge j = k \wedge k = 0; \\
 i \wedge j = k;\;\;j \wedge k = i;\;\;k \wedge i = j \\
  \\
 \end{array}}
\]
(es decir, en la sucesi\'{o}n ($i,j,k,i,j, \cdots$) el producto de
dos vectores sucesivos da el que le sigue):
\[
j \wedge i =  - k,\,k \wedge j =  - i;\,i \wedge k =  - j
\]

\begin{figure}[htb]
\centering
FIGURA
%\includegraphics[scale=0.5]{ijk.eps}
\end{figure}

\textbf{9) }De 4.6), 4.7) y 4.8) resulta que si $v = a_{1}^{} i +
a_{2}^{} j + a_{3}^{} k$ y $w = b_{1}^{} i + b_{2}^{} j + b_{3}^{}
k$.
 entonces:
\[
v \wedge w = \left( {a_{2}^{} b_{3}^{}  - a_{3}^{} b_{2}^{} }
\right)i - \left( {a_{1}^{} b_{3}^{}  - a_{3}^{} b_{1}^{} }
\right)j + \left( {a_{1}^{} b_{2}^{}  - a_{2}^{} b_{1}^{} }
\right)k.
\]
Obs\'{e}rvese que este resultado puede recordarse pensando en el
desarrollo por la primer fila de un determinante:
\[
\left| {{\begin{array}{*{20}c}
  {i} \hfill & {j} \hfill & {k} \hfill  \\
  {a_{1}^{} } \hfill & {a_{2}^{} } \hfill & {a_{3}^{} } \hfill  \\
  {b_{1}^{} } \hfill & {b_{2}^{} } \hfill & {b_{3}^{} } \hfill  \\
\end{array} }} \right| = \left( {a_{2}^{} b_{3}^{}  - a_{3}^{} b_{2}^{} }
\right)i - \left( {a_{1}^{} b_{3}^{}  - a_{3}^{} b_{1}^{} }
\right)j + \left( {a_{1}^{} b_{2}^{}  - a_{2}^{} b_{1}^{} }
\right)k
\]
Esta expresi\'{o}n del producto vectorial puede tambi\'{e}n
tomarse como su definici\'{o}n. M\'{a}s precisamente, dada una
terna ortonormal cualquiera \{\textit{i,j,k}\} (positiva o
negativa), puede definirse el producto de dos vectores por la
expresi\'{o}n dada en 4.9). Se comprueba entonces sin dificultad
que el vector:
 $u = \left( {a_{2}^{} b_{3}^{}  - a_{3}^{} b_{2}^{} } \right)i - \left(
{a_{1}^{} b_{3}^{}  - a_{3}^{} b_{1}^{} } \right)j + \left(
{a_{1}^{} b_{2}^{}  - a_{2}^{} b_{1}^{} } \right)k$ verifica
$\left\| {u} \right\| = \left\| {v} \right\|.\left\| {w}
\right\|.sen\,\theta $ y $u.v = u.w = 0$. Si adem\'{a}s $
\{i,j,k\}$ es una terna positiva puede verificarse, que $
\{v,w,u\}$ es tambi\'{e}n positiva. Luego $u = v \wedge w$ , por
lo tanto esta definici\'{o}n de $v \wedge w$ coincide con la
inicial.
\bigskip

\subsection{Aplicaciones geom\'{e}tricas.}

\textbf{a ) }\textbf {Distancia de un punto a una recta:} sea r
una recta dada por un punto A y un vector \textit{v}. La distancia
de Q a r es:
\[
d\left( {Q,r} \right) = \left| {d\left( {Q,A} \right).sen\,\theta
} \right| = \left\| {AQ} \right\|.sen\,\theta  = \frac{{\left\|
{AQ \wedge v} \right\|}}{{\left\| {v} \right\|}}
\]


\begin{figure}[htb]
\centering
FIGURA
%\includegraphics[scale=0.4]{puntorecta.eps}
\end{figure}
%\begin{center}
%% Warning: EPS-printer was not specified  -- Figure omitted!
%
%\end{center}
%
%
%\bigskip
%
%\begin{center}
%FIGURA 5.1
%\end{center}


\bigskip

Si $A = ( x_{o} ,y_{o} ,z_{o});\,\,Q = (\overline {x} ,\overline
{y} ,\overline {z} );\,v = ai\, + \,bj\, + \,ck$ entonces la
distancia es $ d(Q,r)=$ $$
\frac{\sqrt{[(\overline{y}-y_{0})c-(\overline{z}-z_{0})b]^{2}+
[(\overline{x}-x_{0})c-(\overline{z}-z_{0})a]^{2}+[(\overline{x}-x_{0})b-(\overline{y}-y_{0})a]^{2}}}
{\sqrt{a^{2}+b^{2}+c^{2}}}$$

\bigskip

\begin{ejemplo}\label{piej5.1}En el punto 3 de este cap\'{\i}tulo se vieron
algunas superficies de revoluci\'{o}n particulares. Sea $r\left\{
\begin{array}{l}
 {x = y} \\
 {z = 0}
 \end{array} \right.$ $C:\left\{ \begin{array}{l}
 {x = 0} \\
 {zy = 1}
 \end{array} \right.$. Hallar la superficie de revoluci\'{o}n de eje r
y generatriz C.

Las ecuaciones son:
\[
\left\{ \begin{array}{l}
 {x_{0}^{}  = 0                                 ,\,\,\,            P_{0}^{}  \in C}
\\
 {z_{0}^{} y_{0}^{}  = 1                  ,\,\,\,                        P_{0}^{}
\in C} \\
 {x - x_{0}^{}  + y - y_{0}^{}  = 0              ,\,\,\,            plano ,\,\,\, \pi ,\,\,\, por
,\,\,\, P_{0}^{}  \bot r} \\
 {2z^{2} + \left( {x - y} \right)^{2} = 2z_{0}^{2}  + \left( {x_{0}^{}  -
y_{0}^{} } \right)^{2} ,\,\,\,  dist\left( {P,r} \right) =
dist\left( {P_{0}^{} ,r} \right)}
 \end{array} \right.
\]
eliminando $x_{0}^{} ,y_{0}^{} ,\,z_{0}^{} $ resulta $\left( {x +
y} \right)^{2}\left( {z^{2} - 2xy} \right) = 1$
\end{ejemplo}

\vspace{.5cm}

b)\textbf{Intersecci\'{o}n de dos planos no paralelos:}

Sea $r:\left\{ \begin{array}{l}
 {ax + by + cz + d = 0} \\
 {a'x + b'y + c'z + d' = 0}
 \end{array} \right.$ una recta (dada como intersecci\'{o}n de dos
planos no paralelos). Supongamos que queremos las ecuaciones
param\'{e}tricas de esa recta, es decir, las de la forma: $x =
x_{0}^{}  + \lambda p,\,\,y = y_{0}^{}  + \lambda q,\,\,z =
z_{0}^{}  + \lambda r$ (o tambi\'{e}n $P = P_{0}^{}  + \lambda
v$). Para esto se necesita hallar un punto $P_{0}^{}  = \left(
{x_{0}^{} ,y_{0}^{} ,z_{0}^{} } \right)$ de la recta y un vector
$v = p\,i + q\,j + r\,k$ de la direcci\'{o}n de r.
\bigskip
Esto se puede hacer sin usar el producto vectorial, resolviendo el
sistema de ecuaciones
\[
\left\{ \begin{array}{l}
 {ax + by + cz + d = 0} \\
 {a'x + b'y + c'z + d' = 0}
 \end{array} \right.
\]
Usando el producto vectorial: tomar $P_{0}^{}  = \left( {x_{0}^{}
,y_{0}^{} ,z_{0}^{} } \right)$ una soluci\'{o}n particular del
sistema de ecuaciones. Si el sistema de coordenadas es ortogonal,
entonces:
 $n = ai + bj + ck\,\,y\,\,n' = a'i + b'j + c'k$ son vectores normales a los
planos, y entonces $n \wedge n'$ es un vector de la
intersecci\'{o}n de ambos planos (luego, est\'{a} en la
direcci\'{o}n de la recta de intersecci\'{o}n de ambos).


\begin{figure}[htb]
\centering
FIGURA
%\includegraphics[scale=0.4]{planos.eps}
\end{figure}
%\begin{center}
%% Warning: EPS-printer was not specified  -- Figure omitted!
%
%\end{center}
%
%\bigskip
%\begin{center}
%FIGURA 5.2
%\end{center}

Luego: $n \wedge n'$$ = \left( {bc' - b'c} \right)i - \left( {ac'
- \,ca'} \right)j + \left( {ab' - ba'} \right)k$ es de la
direcci\'{o}n de r.

\bigskip
\textbf{c ) }\textbf {Distancia entre dos rectas:} Se define la
distancia $d\left( {r,r'} \right)$ entre dos rectas $r,r'$ como el
m\'{\i}nimo de $d\left( {P,P'} \right)$ donde $P \in r$ y $P' \in
r'$.

Es claro que este m\'{\i}nimo se obtiene cuando la recta $PP'$ es
perpendicular al mismo tiempo a $r $ y a $r'$. Sea $ r $ dada por
un punto $A \in r$ y un vector $v$  de su direcci\'{o}n y $r'$ por
$B \in r$ y un vector $w$.


Para tener $d\left( {r,r'} \right)$: si \textit{r} y \textit{r'}
son paralelas: basta tomar el punto A y hallar $dist\left( {A,r'}
\right)$; si $r$ y $r'$ se cortan: $d\left( {r,r'} \right)$= 0.


\begin{figure}[htb]
\centering
FIGURA%
FIGURA%\includegraphics[scale=0.5]{rectarecta.eps}
\end{figure}
%\begin{center}
%% Warning: EPS-printer was not specified  -- Figure omitted!
%
%\end{center}
%\bigskip
%\begin{center}
%FIGURA 5.3
%\end{center}

\bigskip

Supongamos ahora que \textit{r} y \textit{r'} no son coplanares.
Si \textit{s} es la perpendicular com\'{u}n , $P_{o} \,\, =
\,\,s\,\, \cap \,\,r^{}$  y  $P_{0}^{'}  = s \cap r'$, entonces:
 $d\left( {r,r'} \right) = d\left( {P_{0}^{} ,P_{0}^{'} } \right) = d\left(
{A,\pi } \right)$ donde $\pi $  es el plano paralelo a \textit{r}
que contiene a \textit{r'}. (ver figura) . Como versor de la
direcci\'{o}n de \textit{s} puede tomarse $n = \frac{{v \wedge
w}}{{\left\| {v \wedge w} \right\|}}$. Luego :
\[
d\left( {r,r'} \right) = d\left( {A,\pi } \right) =
\frac{{1}}{{\left\| {v \wedge w} \right\|}}\left| {\left( {v
\wedge w} \right).\overrightarrow{AB} } \right|
\]


\begin{figure}[htb]
\centering
FIGURA
%\includegraphics[scale=0.5]{normalcomun.eps}
\end{figure}
%\begin{center}
%% Warning: EPS-printer was not specified  -- Figure omitted!
%
%\end{center}
%
%\bigskip
%\begin{center}
%FIGURA 5.3
%\end{center}
\bigskip

\textbf{d ) }\textbf {Perpendicular com\'{u}n a dos rectas:} Si
las dos rectas son paralelas, una perpendicular com\'{u}n es la
recta perpendicular a una de ellas trazada por un punto de la
otra. Si se interceptan , el problema es tambi\'{e}n f\'{a}cil de
resolver. Supongamos ahora que las dos rectas no son coplanares.
La perpendicular com\'{u}n $s$ est\'{a} en el plano $\pi $ de $s$
y $r$, y en el $\pi $' determinado por $s$ y $r'$. Luego $s = \pi
\cap \pi '$. Dada la direcci\'{o}n \textit{v }de \textit{r} y la
$w$ de \textit{r'}, y $A \in r,\,\,B \in r'$, el plano $\pi $
queda determinado por $A$,\textit{v} y $v \wedge w$. El $\pi $'
por B,\textit{w},$v \wedge w$, pues $v \wedge w$ es un vector de
la direcci\'{o}n de \textit{s}. Las ecuaciones de $\pi $ y $\pi $'
as\'{\i} obtenidas constituyen un par de ecuaciones que
representan la recta \textit{s}.
\bigskip

\textbf{e ) }\textbf{Volumen de un tetraedro: }Consideremos un
tetraedro dado por sus v\'{e}rtices $A, B, C, D$. Su volumen es $V
= \frac{{1}}{{3}}$ \'{a}rea de la base $ \times $ altura.
Tendremos: \'{a}rea base = $1/2\left\| {\left( {B - A} \right)
\wedge \left( {C - A} \right)} \right\|$.

\begin{figure}[htb]
\centering
FIGURA
%\includegraphics[scale=0.4]{tetra.eps}
\end{figure}

%% Warning: EPS-printer was not specified  -- Figure omitted!
%
%\bigskip
%\begin{center}
%FIGURA 5.4
%\end{center}

 Si \textit{n} es el versor normal a la base, la altura es $
 \left| {n.\left( {D - A} \right)} \right|$\, con \,$n = \frac{{\left(
{B - A} \right) \wedge \left( {C - A} \right)}}{{\left\| {\left(
{B - A} \right) \wedge \left( {C - A} \right)} \right\|}}.$ Luego
$V = 1/6.\left\| {\left( {B - A} \right) \wedge \left( {C - A}
\right)} \right\|.\frac{{\left| {\left[ {\left( {B - A} \right)
\wedge \left( {C - A} \right)} \right].\left( {D - A} \right)}
\right|}}{{\left\| {\left( {B - A} \right) \wedge \left( {C - A}
\right)} \right\|}}$
\[ {\rm por\,\,\, lo\,\,\, que\,\,\,\,}
V = 1/6.\left| {\left[ {\left( {B - A} \right) \wedge \left( {C -
A} \right)} \right].\left( {D - A} \right)} \right|
\]

\bigskip

\subsection{Producto mixto}
\bigskip
Es una operaci\'{o}n definida de $V\, \times \,\,V\,\, \times
\,\,V\,$ en $\mathbb{R}$. Dados tres vectores $u,v, \,y\,\,w$,
llamamos producto mixto al n\'{u}mero $\left( {v \wedge w}
\right).u$ que indicamos $v \wedge w.u$, o tambi\'{e}n $u.\left(
{v \wedge w} \right)$.
\bigskip
Consideremos un sistema ortonormal $\left\{ {i,j,k} \right\}$ y
sean:
\[
v = a_{1}^{} i + a_{2}^{} j + a_{3}^{} k, w = b_{1}^{} i +
b_{2}^{} j + b_{3}^{} k, u = c_{1}^{} i + c_{2}^{} j + c_{3}^{} k.
\]
Entonces:
\[
v \wedge w = \left| {{\begin{array}{*{20}c}
  {a_{2}^{} } \hfill & {a_{3}^{} } \hfill  \\
  {b_{2}^{} } \hfill & {b_{3}^{} } \hfill  \\
\end{array} }} \right|i - \left| {{\begin{array}{*{20}c}
  {a_{1}^{} } \hfill & {a_{3}^{} } \hfill  \\
  {b_{1}^{} } \hfill & {b_{3}^{} } \hfill  \\
\end{array} }} \right|j + \left| {{\begin{array}{*{20}c}
  {a_{1}^{} } \hfill & {a_{2}^{} } \hfill  \\
  {b_{1}^{} } \hfill & {b_{2}^{} } \hfill  \\
\end{array} }} \right|k.
\]
\[
v \wedge w.u = \left| {{\begin{array}{*{20}c}
  {a_{2}^{} } \hfill & {a_{3}^{} } \hfill  \\
  {b_{2}^{} } \hfill & {b_{3}^{} } \hfill  \\
\end{array} }} \right|c_{1}^{}  - \left| {{\begin{array}{*{20}c}
  {a_{1}^{} } \hfill & {a_{3}^{} } \hfill  \\
  {b_{1}^{} } \hfill & {b_{3}^{} } \hfill  \\
\end{array} }} \right|c_{2}^{}  + \left| {{\begin{array}{*{20}c}
  {a_{1}^{} } \hfill & {a_{2}^{} } \hfill  \\
  {b_{1}^{} } \hfill & {b_{2}^{} } \hfill  \\
\end{array} }} \right|c_{3}^{} =
\left| {{\begin{array}{*{20}c}
  {c_{1}^{} } \hfill & {c_{2}^{} } \hfill & {c_{3}^{} } \hfill  \\
  {a_{1}^{} } \hfill & {a_{2}^{} } \hfill & {a_{3}^{} } \hfill  \\
  {b_{1}^{} } \hfill & {b_{2}^{} } \hfill & {b_{3}^{} } \hfill  \\
\end{array} }} \right|
\]
\begin{center}
(Por desarrollo por la primer fila de ese determinante)
\end{center}
Usando el hecho de que si se permutan dos filas de una matriz
entre s\'{\i} el determinante s\'{o}lo cambia de signo, resulta
que dos de esas permutaciones no cambian el determinante; luego:
\[
\left| {{\begin{array}{*{20}c}
  {c_{1}^{} } \hfill & {c_{2}^{} } \hfill & {c_{3}^{} } \hfill  \\
  {a_{1}^{} } \hfill & {a_{2}^{} } \hfill & {a_{3}^{} } \hfill  \\
  {b_{1}^{} } \hfill & {b_{2}^{} } \hfill & {b_{3}^{} } \hfill  \\
\end{array} }} \right| = \left| {{\begin{array}{*{20}c}
  {b_{1}^{} } \hfill & {b_{2}^{} } \hfill & {b_{3}^{} } \hfill  \\
  {c_{1}^{} } \hfill & {c_{2}^{} } \hfill & {c_{3}^{} } \hfill  \\
  {a_{1}^{} } \hfill & {a_{2}^{} } \hfill & {a_{3}^{} } \hfill  \\
\end{array} }} \right| = \left| {{\begin{array}{*{20}c}
  {a_{1}^{} } \hfill & {a_{2}^{} } \hfill & {a_{3}^{} } \hfill  \\
  {b_{1}^{} } \hfill & {b_{2}^{} } \hfill & {b_{3}^{} } \hfill  \\
  {c_{1}^{} } \hfill & {c_{2}^{} } \hfill & {c_{3}^{} } \hfill  \\
\end{array} }} \right|
\]
En consecuencia: $\left( {v \wedge w} \right).u = \left( {u \wedge
v} \right).w = \left( {w \wedge u} \right).v$. Es decir, que en la
sucesi\'{o}n (\textit{u,v,w,u,v,w}) el producto vectorial de dos
vectores sucesivos multiplicado escalarmente por el siguiente, da
lo mismo cualesquiera sean los tres vectores sucesivos. Se puede
entonces hablar del producto mixto de \textit{u, v, w} sin indicar
si se trata de $\left( {u \wedge v} \right).w$ o de $u.\left( {v
\wedge w} \right)$, pues el resultado es el mismo. Es por esto que
se escribe (\textit{u,v,w}) como notaci\'{o}n para $\left( {u
\wedge v} \right).w = u.\left( {v \wedge w} \right)$.

Observamos que (\textit{v,w,u}) = 0 si y s\'{o}lo si
\'{a}ng$\left( {v \wedge w,u} \right) = \pi /2$ o alguno de los
vectores es el nulo. Como \'{a}ng $\left( {v \wedge w,v} \right) =
\,\,\mbox{\'{a}ng}\,\,\left( {v \wedge w,w} \right) = \pi /2$ ,
para que (\textit{v,w,u}) = 0 es necesario y suficiente que
\textit{v, w} y \textit{u} sean coplanares. (Obs\'{e}rvese que
esto podr\'{\i}a sacarse como conclusi\'{o}n del c\'{a}lculo del
volumen del tetraedro. $(v,w,u) = 0$ si y s\'{o}lo si el volumen
del tetraedro determinado por esos vectores es 0, esto es
equivalente a decir que los vectores son coplanares).
\bigskip

Esta condici\'{o}n permite escribir la ecuaci\'{o}n vectorial del
plano dado por tres puntos $A, B, C$ en otra forma. Decir que $P$
pertenece a ese plano equivale a decir que los vectores $P-A, B-A$
y $C-A$ son coplanares, o sea $(P-A,B-A,C-A) = 0$.


Pasando a coordenadas, si \textit{P=}(x,y,z),\textit{A=}($a_{1}^{}
,a_{2}^{} ,a_{3}^{} $), $B = \left( {b_{1}^{} ,b_{2}^{} ,b_{3}^{}
} \right)$ y $C = \left( {c_{1}^{} ,c_{2}^{} ,c_{3}^{} } \right)$
esta ecuaci\'{o}n se escribe:
\[
\left| {{\begin{array}{*{20}c}
  {x - a_{1}^{} } \hfill & {y - a_{2}^{} } \hfill & {z - a_{3}^{} } \hfill
\\
  {b_{1}^{}  - a_{1}^{} } \hfill & {b_{2}^{}  - a_{2}^{} } \hfill &
{b_{3}^{}  - a_{3}^{} } \hfill  \\
  {c_{1}^{}  - a_{1}^{} } \hfill & {c_{2}^{}  - a_{2}^{} } \hfill &
{c_{3}^{}  - a_{3}^{} } \hfill  \\
\end{array} }} \right| = 0,\,\,\,\,
 \mbox{\'{o}}\,\, \mbox{tambi\'{e}n:}\,\,\, \left| {{\begin{array}{*{20}c}
  {x} \hfill & {y} \hfill & {z} \hfill & {1} \hfill  \\
  {a_{1}^{} } \hfill & {a_{2}^{} } \hfill & {a_{3}^{} } \hfill & {1} \hfill
\\
  {b_{1}^{} } \hfill & {b_{2}^{} } \hfill & {b_{3}^{} } \hfill & {1} \hfill
\\
  {c_{1}^{} } \hfill & {c_{2}^{} } \hfill & {c_{3}^{} } \hfill & {1} \hfill
\\
\end{array} }} \right| = 0.\]
Para ver esto obs\'{e}rvese que si se le resta la 2da. fila a la
1ra., la 3da. y la 4ta. filas, el  determinante no cambia.
As\'{\i} se tiene:
\[
\left| {{\begin{array}{*{20}c}
  {x - a_{1}^{} } \hfill & {y - a_{2}^{} } \hfill & {z - a_{3}^{} } \hfill &
{0} \hfill  \\
  {a_{1}^{} } \hfill & {a_{2}^{} } \hfill & {a_{3}^{} } \hfill & {1} \hfill
\\
  {b_{1}^{}  - a_{1}^{} } \hfill & {b_{2}^{}  - a_{2}^{} } \hfill &
{b_{3}^{}  - a_{3}^{} } \hfill & {0} \hfill  \\
  {c_{1}^{}  - a_{1}^{} } \hfill & {c_{2}^{}  - a_{2}^{} } \hfill &
{c_{3}^{}  - a_{3}^{} } \hfill & {0} \hfill  \\
\end{array} }} \right|\,\, = \,\,
\left| {{\begin{array}{*{20}c}
  {x - a_{1}^{} } \hfill & {y - a_{2}^{} } \hfill & {z - a_{3}^{} } \hfill
\\
  {b_{1}^{}  - a_{1}^{} } \hfill & {b_{2}^{}  - a_{2}^{} } \hfill &
{b_{3}^{}  - a_{3}^{} } \hfill  \\
  {c_{1}^{}  - a_{1}^{} } \hfill & {c_{2}^{}  - a_{2}^{} } \hfill &
{c_{3}^{}  - a_{3}^{} } \hfill  \\
\end{array} }} \right| = 0.
\]

%\end{document}
