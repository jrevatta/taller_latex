\documentclass [a4paper,12pt]{article}
\usepackage{epsfig,amsmath,amsthm,amsfonts,amssymb}
\usepackage{graphicx,epsfig,epsf}
\usepackage[spanish]{babel} % Manejo de idiomas
\usepackage{hyperref}
\usepackage[latin1]{inputenc} % Escritura en castellano con acentos
%Escritura tipo Caligraf�a
\usepackage{calligra}
%%%%%%%%%%%%%%%%%%%%%%%%%%%%%%%%%%%%%%%%%%%%%%%%%%%%%%%
\newtheorem{theorem}{Teorema}
\newtheorem{lemma}{Lema}
\newtheorem{proposition}{Proposici�n}
\newtheorem{corollary}{Corolario}
\newtheorem{assumption}{Suposici�n}
\newtheorem{definition}{Definici�n}
\newtheorem{example}{Ejemplo}
\newtheorem{note}{Observaci�n}
\newtheorem{exercise}{Ejercicio}
\newtheorem{remark}{Observaci\'on}

\usepackage[T1]{fontenc} % Escritura en castellano con acentos

\usepackage{times} % Fuente de letras

\begin{document}
\begin{figure}[htbp]
\scalebox{0.25}{\includegraphics{oficial_450.ps}}
\end{figure}
\vspace{-2.5cm}
\begin{center}
{\Large Universidad Nacional Mayor de San Marcos}\\
{\calligra Universidad de Per�, Decana de Am�rica}\\
Facultad de Ciencias Matem�ticas\\
E.A.P. Computaci�n Cient�fica\\
\end{center}
\begin{center}
\textbf{An�lisis Funcional II}\\
\textbf{Ex�men Parcial}
\end{center}
\vskip1cm
\begin{enumerate}

\item
\textbf{5 puntos}\\
Enuncie y demuestre alguna de las versiones del Teorema de Hanh Banach (Espacios Vectoriales o de Banach).

\item

Sea $X$ un espacio normado sobre $\mathbb{K}$. Demuestre que:
\begin{enumerate}
\item
si $L$ un subespacio vectorial de $X$ entonces para todo $u_0 \in X$ con
\[ dist(u_0,L) >0 \]
existe una funcional lineal continua $f:X \rightarrow \mathbb{K}$ tal que
\[f(u) = 0, \hskip.5cm \mbox{para todo} \hskip.5cm u \in L \]
con $\|f\| =1$ y $f(u_0)=dist(u_0,L)$.  \textbf{6 puntos}
\item
Si $u \in X$ y 
\[f(u)= 0  \hskip1cm \forall f \in X^* \]
entonces $u=0$. \\
\textbf{4 puntos}
\end{enumerate}


\item
\textbf{5 puntos}
Sean $A$ y $B$ conjuntos convexos no vacios en el espacio normado real $X$. Usando el Teorema de Hanh Banach, demuestre que
\begin{enumerate}
\item
$A$ y $B$ pueden ser estrictamente separados por un hiperplano provisto de las condiciones
\[B \bigcap Int(A) = \phi  \hskip1cm Int(A)\neq \phi\]
\item
$A$ y $B$ pueden ser separados por un hiperplano cerrado provisto de las condiciones
\[B \bigcap A = \phi \]
y ambos son conjuntos abiertos.
\end{enumerate}

\end{enumerate}



\vskip1cm  RLC \hskip8cm Sem 2010-II

\end{document}